\section{Introduction}

Syncretism is the combination of different beliefs from different cultures into a new system of beliefs in a culture. This essay will explore how views on homosexuals evolved, from native pre-colonial views to the manifestation of European homophobia after colonisation and explore how colonial policies have affected the predominant views on homosexual activity in the area we now call China. I will use primary sources from the period, including folk stories about homosexual activities, and official histories that talk about the male “lovers” of various high officials and emperors. Most of this essay will be focused on secondary sources, research papers and books that go into views on homosexual activities from China from the imperial era into the \nth{20} century. I will go through how homophobia evolved and how that fits into our model of anthropology.

The development of homophobia in China is mainly the result of European colonisation and the syncretism of native Chinese and external European cultural philosophies.

\newpage
\section{Pre-colonial China}

Homosexual relationships have existed and been documented in China since the Zhou Dynasty, back in the bronze age\autocite{hinsch1990passions}. The Chinese had always had a completely different view of homosexuality and homosexual acts.

The Chinese before colonisation did not have a concept of homosexuality, instead, they viewed homosexual acts as actions and preferences, not as some innate part of who you are. You were not bisexual in imperial china, you preferred having sex with men\Autocite{1997後殖民同志}. Preferring men were seen as just the same as preferring a skinnier person, or preferring a fatter person. Preferences may change throughout the ages, but there will always be people whose preferences are different to the majority. Sexual preferences were just like any other preference, as long as they don't interfere with your other duties as a son, as a subject, and as a member of society, you're free to pursue any sexual desires you pleased. Of course, most of that only applies to men, mostly of the aristocracy\autocite{hinsch1990passions}.

In Imperial China, a man has a duty to their family, and as a part of that duty, one has to give birth to sons, as to extend their family lineage, and marriage was an important part of that. Marriage was also not about the individuals but about their families, and whether of not those individuals were in love, or were even sexually attracted to each other played very little importance to the families as long as they produced male children. Often the men were allowed to indulge in their sexual desires outside of marriage with no objection from their wife or anyone else\Autocite{10.2307/2961796}. The fulfillment of those sexual desires included sleeping with other men, marrying concubines(in imperial China, one man could only have one official wife but can take on as many concubines as they want, the children of concubines were of lower status than the children of the official wife), and sleeping with prostitutes, which included both males and females. Back in the Western Han dynasty, the height homosexual activity in China, most of the emperors had at least one male partner, despite having a wife, the empress, as well as countless concubines\Autocite{hinsch1990passions}.

Homosexual acts were often not done two members of the same social class, very often there exists a hierarchical difference between the two participants with one partner being of higher social status than the other, even though this was not always the case\Autocite{huang_2013_malemale}. We know very little of the activities of members of the lower class, as only the activities of the aristocracy were deemed to be important enough to be recorded, and in the far past, back in the Han and Zhou dynasties, only irregular activities were recorded. The hierarchical nature of the homosexual relationships was often reflected in the sex itself, the more dominant member of the relationship socially was often the more dominant member sexually, essentially acting as a dom, whereas the other member takes on a more submissive role, essentially a sub. It often also meant the more dominant person also acted as the penetrator, or the top, and the submissive member acting as the penetrated, or the bottom\autocite{huang_2013_malemale}.

When one partner was of significantly higher class than the other, the partner with the lower social status was often given many gifts from the other partner\Autocite{hinsch1990passions}, which may even be titles and ranks if one of the partners was the emperor, as is the case with Dong Xian(董贤/董賢). Dong Xian was the favourite of the Ai Emperor of the Han Dynasty, he was given the title of Marquis of Gao'an. His family also received many gifts including a title of marquis for his father. One day, when Emperor Ai got up to work, Dong Xian was still sleeping on Emperor Ai's sleeves. As to not disturb Dong Xian, Emperor Ai cut his sleeve off. This is where the term \textit{cut sleeve}, which is a common phrase referring to gay people in China, comes from. The power granted to Dong Xian was so advanced that he essentially held the reins to the empire. Before Emperor Ai died, he handed the Imperial Seals to Dong Xian declaring Dong Xian to be emperor, but this was not accpeted by his political enemies and Dong Xian having not consolidated any power, was forced to commit suicide, with a relative of the empress instating a puppet child emperor before taking the throne for himself, ending the Western Han Dynasty.

Homosexual partners may sometimes be compared to heterosexual couples\Autocite{hinsch1990passions}, but often these relationships were treated very differently. Being in the submissive partner in a male-male relationship may bring on different responsibilities depending on the exact nature of the relationship. Some relationships may be quite casual and temporary, with the more dominant and older partner helping the younger partner find a spouse when they turn of age(20) as is the case with many \textit{shu tong}(书童/書童). It may also be that the submissive partner takes on the social status and responsibilities of a woman and were expected to remain faithful to the dominant partner as a woman was expected to remain faithful to their husband. The Chinese did not find this double standard between men and women, or between social higher and social subservient contradictory as a result of the patriarchy in China.

Partaking in homosexual acts, especially as the dominant, wasn't seen as feminine, as is often seen in the west, instead, it was just a normal part of a man's life.

Submissivity in male-male sexual activity became more and more associated with femininity through the ages\Autocite{hinsch1990passions}. Submissivity became less and less accepted by members of the upper class and became further associated with the lower classes. In \citetitle{xueqin2009dream}, one of the characters who comes from a noble background who has fallen into hard times, when confronted by someone who wishes to have sex with him, he gets offended and beats up the other person, not because he is offended by insinuating he is gay, but by insinuating he is the submissive one, which he takes great offence to as he associates that with being female or feminine and of lower status.

Class differences was an important part in the sexual acts that one participated in. The amount of class difference resulted in the differences between the positions. When there is a larger class difference, the sexual positions that one is allowed to participate in is more rigid than when there is only a minimal difference in class. This remained an important part of homosexual activity within China from the first written records of male-male sexual activity to the end of the imperial era in China with the fall of the Qing dynasty.

Male-male sex when there is a big class difference may even lead to differences in gender roles. Eunuchs were often preferred over normal men in a sexually submissive because of their inability to engage in the penetration that is a mark of domination in male-male sex. Eunuchs were often not treated as being “full” men as they were not able to reproduce. They were often seen as more feminine as their lack of testosterone made their features appear more traditionally feminine, though whether or not the Eunuch saw themself as feminine depends on the individual.

Gender and sexuality often get combined in many folk stories. In one folk story, a prostitute was “rescued” by a rich noble, in the story, he was expected to be faithful and loyal to his “saviour” and was not supposed to have sex with anyone else, which is generally associated with female wives, whilst the rich noble was allowed go around and have sex with anyone they wanted without any problem, as with male husbands traditionally are able to. Another story tells of a poor boy whose family is financially supported in exchange for his body, he even gets castrated in order to further please his “lover”. After his “lover” dies, he starts crossdressing as a woman, even living as a widow, bring up his “lover's” son as a mother, bringing him up to adulthood who eventually passes the civil examinations to become a high official, and is praised for being a good Confucian mother. These stories, though probably not completely true, show how in China during this period, gender and sexual position, as well as class is inherently linked.

Partaking in homosexual activity in a dominant position got associated with the upper class as submissiveness became associated with prostitution and the socially inferior\Autocite{hinsch1990passions}.

Homosexual activities between two men and homosexual activities between two women were not seen as the same\Autocite{hinsch1990passions}. The Chinese did not view the two genders as the same when it comes to sexuality as was the way in Europe.

\section{Colonisation}

The concept of homosexuality was brought to China during the Opium wars during the colonial age when Europeans defeated the Qing Army and established colonial port cities. The Chinese finally saw that their way of society and government was at threat from an even larger power, with a completely different way of life from another part of the world, the first time in history. Westernisation movements started to spring up in the late Qing Dynasty, after the humiliating defeat by the British in the Opium Wars. Western philosophy and science started to be seen as progressive and much of the old Chinese way was seen as backward. These movements started in the late \nth{19} century and continued into the \nth{20} as the Qing Empire fell and a new republic taking its place. The progressives in this new Society saw Homosexual activity as backwards and a part of the old way, as homosexuality was seen by western psychologists as a mental disease and something to be rid of\Autocite{1997後殖民同志}. But as science and civil rights progressed through the late \nth{20} century, these developments were not accessible to the people within mainland China as the Communist government took over the mainland. The Attitude towards the West also started to change from an attitude of admiration, to an attitude of resentment. Through this era, the Chinese started to forget the homosexual history of China, as historical records were lost, destroyed, or forgotten\Autocite{1997後殖民同志}. The literary movement of the early \nth{20} century converted the standard written language from Classical Chinese, based on the language of the Han dynasty 2000 years ago, to a new written language much more based in the vernacular language of the day. This shift made reading and writing much easier for the Chinese public, as they no longer needed to learn a whole new system of Grammar and Vocabulary just to communicate with writing(if you are lucky enough to be speaking one of those “civilized dialects” that the new standard is based on), but it makes it much harder to access the 2000 years of history that was all written in Classical Chinese. Records of homosexual activity were just written there in plain Classical Chinese, but as only the few could fully read and understand, they were free to include and exclude as much as they want in the teaching of history. Eventually, the Chinese forgot their history of homosexuality, with nothing being taught in schools about homosexuality in history, and homosexuals remaining hidden through Chinese Society as homophobia dominated, the new generations of Chinese People grew up with no knowledge of homosexuality, and thinking that it has always been this way. Homosexual acceptance had to be brought back from the west through the last two decades of the \nth{20} century and the \nth{21}, with new western ways of thinking about sexuality.

During the colonial era, Chinese people started to move out of China and into western countries, bringing with them their tradition of homosexuality, the homophobic west were not very keen on the idea male-male sex, the Spanish blamed the Chinese for spreading homosexuality to the native Philipinos\autocite{10.2307/42632420}.

Homosexual rape was a very severe issue in the Qing Dynasty, and Europeans took this as more of a sign of Chinese moral degeneracy. The Europeans equated the rapes with homosexuality itself, and with many officials in China engaging in homosexual activities, the Europeans saw those officials all as rapists\Autocite{doi:10.1300/J236v07n01_08}.

Western Homophobia compounded with strict Neoconfucian values that has risen through the late Qing Dynasty. The new interpretations of Confucianism were more strict on extra-marital sex and with such, it combined with Western homophobia, as it became no longer moral to have sex outside marriage, and with the purpose of marriage being to have children, sex between two men was no longer seen as moral. This combined with western science classifying homosexuality as a disease, making homosexuality less and less accepted.

At the start of the Westernisation movements, the main focus was to import western technology and rebuild the Empire with Western Technology backed with Chinese Philosophies and morals. As time went on however, the Chinese got repeatedly defeated by the western armies and the sentiment began to change. Chinese Values were starting to be seen as backwards and detrimental to the Chinese Society, and Western Philosophies and moral traditions needed to be adapted to keep up with the West.

Hong Kong as a British colony, was stuck right in between China and the west. Western sentiments were particularly strong in this city, as well as others that were under heavy colonial influence. Hong Kong, as a British colony followed laws of the British Empire, which included laws that prosecuted people for engaging in homosexual acts. Western Philosophies and morals heavily penetrated the fabrics of society in Hong Kong, western was often seen as superior to Chinese in almost everything including the view on sexuality. Gradually, homosexuality began to be associated with Europeans, and many believe that homosexuality is exclusive to Europeans and that Chinese do not participate in it. People have started to associate homophobia with Chinese Culture, when if you actually look back into history, it is the Europeans that brought homophobia to China, with it having no place in historical Chinese society.

The effects of Westernisation continued well past colonisation itself. Even after China abandoned the west in favour of the Soviet Union, Homophobia continued within Communist China to this day. Despite being the most populous state on Earth, with the biggest LGBT population, LGBT people remain largely invisible in society. Though it is no longer illegal to be LGBT in China, members of the community still face stigma and discrimination by family members, community members, and the state\Autocite{wang2020mapping}. All of this is a result of colonisation and homophobia that has been integrated into the Confucian cultural base of China from European homophobia. Homosexuals within China are still facing increased stigma and discrimination related to HIV or homosexuality in general.

The acceptance of homophobia was partly a result of China resisting the imperial past and embracing the western ideals of science. China clamped on to these ideas by isolating itself from newer western science in the latter half of the \nth{20} century, when western science turned around on the idea of homosexuality being a sin. Only until recently has the acceptance of homosexuality slowly crawl back into China.

\section{Analysis}

The Chinese cultural perceptions of gender and sexuality before colonisation had similarities compared to European notions of gender and sexuality. Both cultures had a highly patriarchal societal structure that limited the rights of women in many different ways. Both had strict religious or philosophical traditions with strict moral codes that puts an emphasis on reproduction and the continuing of family lines. But the two societies had very different ideas surrounding extra-marital sex and homosexual activities.

The way a culture defines and interacts with sexualities is often based on cultural norms and values. Both the Chinese and the Europeans had a strong emphasis on the importance of reproduction, which is not possible through homosexual sex, but the two cultures had vastly different ideas on marriage. The Chinese saw marriage as a purely familial issue, where its only purposes were to have children and to strengthen the bond between two families, whether politically or socially, thus it was fine for the male to engage in sex outside of marriage for pleasure, which naturally included sex with other males. The Chinese society was highly patriarchal and did not treat the women in the same way as they treated men. Women in a marriage were not allowed to have sex with another man outside their marriage, with female-female being rare. The Europeans viewed homosexuality as a sin.

In China, the development of Neoconfucian philosophies throughout the Ming and Qing dynasties combined with the stricter structure of sex of the Mongols and the Manchus, who had a significant influence on the society and culture of China, resulting in stricter cultural regulations on homosexuality, though this was never able to completely remove homosexuality from daily life in China.

During the Colonisation era, China fell behind Europe in wealth, power, and technology, leading to the exploitation of the Qing dynasty through unfair treaties. The British won two opium wars that forced China to allow the sale of opium within the Qing dynasty. Other European powers soon followed with the exploitation of the Qing dynasty. This was the first time in millennia that a completely non-Chinese civilisation was able to dominate a China. Previous dynasties that have been ruled by an external ethnic group such as the Yuan and the Qing dynasties have always eventually assimilated into the wider Han Chinese culture, changing it slightly along the way.

Colonisation made the Chinese people feel inferior technologically, then morphing into feeling inferior culturally. The Chinese started to incorporate European religious homophobia into the native philosophy of Neoconfucianism which had already been developing its own anti-homosexual beliefs. The Neoconfucians believed in a stricter moral policy against extramarital sex, which included most of the homosexual sex that was practised within China. During Colonisation, they started blaming the rampant failing of morals for the failing of the Chinese state and the Chinese culture. They started adopting western ideas around homosexuality, including that it is a sin against nature. These sentiments eventually passed through the entire Chinese society.

At the end of the \nth{19} and the start of the \nth{20} century, Chinese society adopted western technologies and sciences allowing China to become an industrial society, but with it came western philosophies, including homophobia. According to the classical model of sociocultural evolution, human societies evolve in a predetermined path that every civilisation will follow. According to that model, China has advanced another step in terms of social progress. But the introduction of homophobia during this period also shows that this model does not adequately show social progress at least when it comes to queer and feminist theories. The modern more accepted theory of sociocultural evolution involves multilinear evolution instead of unilinear evolution. Homophobia in China is further evidence for multilinear evolution when it comes to social progress.

Syncretism often occurs in positions where one group has significant influence on another group. Syncretism depends on the amount of a culture that was kept through the process of cultural interaction as well as the amount of influence of the other culture. The Chinese were able to keep a significant amount of their culture, but the cultural dominance of European powers were significant and many aspects of the Chinese culture was lost through both external and internal pressures, through the fall of the Qing dynasty into the communist led Cultural Revolution in the 1960s and 70s.

With regard to homosexuals, syncretism led to the development of homophobia among the Chinese public. Western homophobia was combined into Neoconfucian beliefs. This syncretism continues to affect Chinese Society to this day

\newpage
\section{Conclusion}

Homophobia within China was mostly a result of European colonisation. European ideas on homosexuality manifested itself on Chinese culture during a time of European dominance in Asia and stayed into the isolation of China from the west after the Chinese Civil War and the victory of the Communists. Homophobia was brought in in an attempt to modernise and westernise China. What resulted is a syncretism of native Chinese Confucianism ideas of marriage and gender roles with European Christian beliefs of homosexuality as unnatural and a sin. This syncretism has largely carried on to this day, even after China started to kick out elements of the west during the Great Leap Forward and the Cultural Revolution.
